% Header
\documentclass{article}\usepackage[]{graphicx}\usepackage[]{color}
%% maxwidth is the original width if it is less than linewidth
%% otherwise use linewidth (to make sure the graphics do not exceed the margin)
\makeatletter
\def\maxwidth{ %
  \ifdim\Gin@nat@width>\linewidth
    \linewidth
  \else
    \Gin@nat@width
  \fi
}
\makeatother

\definecolor{fgcolor}{rgb}{0.345, 0.345, 0.345}
\newcommand{\hlnum}[1]{\textcolor[rgb]{0.686,0.059,0.569}{#1}}%
\newcommand{\hlstr}[1]{\textcolor[rgb]{0.192,0.494,0.8}{#1}}%
\newcommand{\hlcom}[1]{\textcolor[rgb]{0.678,0.584,0.686}{\textit{#1}}}%
\newcommand{\hlopt}[1]{\textcolor[rgb]{0,0,0}{#1}}%
\newcommand{\hlstd}[1]{\textcolor[rgb]{0.345,0.345,0.345}{#1}}%
\newcommand{\hlkwa}[1]{\textcolor[rgb]{0.161,0.373,0.58}{\textbf{#1}}}%
\newcommand{\hlkwb}[1]{\textcolor[rgb]{0.69,0.353,0.396}{#1}}%
\newcommand{\hlkwc}[1]{\textcolor[rgb]{0.333,0.667,0.333}{#1}}%
\newcommand{\hlkwd}[1]{\textcolor[rgb]{0.737,0.353,0.396}{\textbf{#1}}}%

\usepackage{framed}
\makeatletter
\newenvironment{kframe}{%
 \def\at@end@of@kframe{}%
 \ifinner\ifhmode%
  \def\at@end@of@kframe{\end{minipage}}%
  \begin{minipage}{\columnwidth}%
 \fi\fi%
 \def\FrameCommand##1{\hskip\@totalleftmargin \hskip-\fboxsep
 \colorbox{shadecolor}{##1}\hskip-\fboxsep
     % There is no \\@totalrightmargin, so:
     \hskip-\linewidth \hskip-\@totalleftmargin \hskip\columnwidth}%
 \MakeFramed {\advance\hsize-\width
   \@totalleftmargin\z@ \linewidth\hsize
   \@setminipage}}%
 {\par\unskip\endMakeFramed%
 \at@end@of@kframe}
\makeatother

\definecolor{shadecolor}{rgb}{.97, .97, .97}
\definecolor{messagecolor}{rgb}{0, 0, 0}
\definecolor{warningcolor}{rgb}{1, 0, 1}
\definecolor{errorcolor}{rgb}{1, 0, 0}
\newenvironment{knitrout}{}{} % an empty environment to be redefined in TeX

\usepackage{alltt}
\usepackage[margin=0.5in]{geometry}
\usepackage{datetime}
\IfFileExists{upquote.sty}{\usepackage{upquote}}{}

\begin{document}
\hrule
\begin{quote}
  \begin{flushleft}
    Patrick R. Gillespie\\
    Walsh Fellow\\
    Teagasc, REDP\\
    Athenry, Co Galway, Ireland\\
    patrick.gillespie@teagasc.ie\\
  \end{flushleft}
\end{quote}
\hrule
\begin{flushleft}
  \textbf{\jobname.Rnw}\\
  \today at \currenttime
\end{flushleft}
\hrule
\vspace{1cm}

\begin{flushleft}
\subsection*{Clearing and loading}
\begin{knitrout}
\definecolor{shadecolor}{rgb}{0.969, 0.969, 0.969}\color{fgcolor}\begin{kframe}
\begin{alltt}
\hlkwd{rm}(list = \hlkwd{ls}())
\hlcom{# paneldir <- '/media/MyPassport/Data/data_NFSPanelAnalysis/'}
paneldir <- \hlstr{"D:\textbackslash{}\textbackslash{}Data/data_NFSPanelAnalysis/"}
\hlkwd{source}(\hlkwd{paste}(paneldir, \hlstr{"code_R/dirstruct.R"}, sep = \hlstr{""}))
\hlkwd{setwd}(Rdir)
\hlkwd{library}(foreign)  \hlcom{#...for the ability to read .dta files}
\hlkwd{library}(data.table)  \hlcom{#... useful for big \hlkwd{datasets} (makes manipulation faster & easier)}
\end{alltt}


{\ttfamily\noindent\color{warningcolor}{\#\# Warning: package 'data.table' was built under R version 2.15.3}}

{\ttfamily\noindent\itshape\color{messagecolor}{\#\# data.table 1.9.2\ \ For help type: help("{}data.table"{})}}\begin{alltt}
\hlkwd{library}(xtable)  \hlcom{#... to get nice looking tables.}
\end{alltt}


{\ttfamily\noindent\color{warningcolor}{\#\# Warning: package 'xtable' was built under R version 2.15.3}}\end{kframe}
\end{knitrout}

 
\subsection*{Data import and set up}

Next code chunk does the following:
\begin{itemize}
  \item {Read the NFS dataset into a data.table (type of data.frame - useful for large datasets) and ID the key
        variables (like subscripts for the data). Sorts by those too.}
  \item {Select only Specialist Dairying}
  \item {Tabulate farm numbers by year}
\end{itemize}
\begin{knitrout}
\definecolor{shadecolor}{rgb}{0.969, 0.969, 0.969}\color{fgcolor}\begin{kframe}
\begin{alltt}
nfs <- \hlkwd{data.table}(\hlkwd{read.dta}(\hlkwd{paste}(nfsdatadir, \hlstr{"nfs_data.dta"}, sep = \hlstr{""})), key = \hlstr{"fffadnsy"})
nfs <- nfs[fffadnsy == 411]
\hlkwd{setkey}(nfs, farmcode, year)  \hlcom{# reset key}
nfs.farms <- nfs[year > 1998 & year < 2008, \hlkwd{length}(farmcode), by = year]
\end{alltt}
\end{kframe}
\end{knitrout}


Then do the following for the FADN data to get a comparable dataset: 
\begin{itemize}
  \item{Load the Irish FADN dataset}
  \item{Select Specialist Dairying (A30==4110)}
  \item{Tabulate farm numbers by year}
\end{itemize}
\begin{knitrout}
\definecolor{shadecolor}{rgb}{0.969, 0.969, 0.969}\color{fgcolor}\begin{kframe}
\begin{alltt}
fadn <- \hlkwd{data.table}(\hlkwd{read.csv}(\hlkwd{paste}(fadndir9907, \hlstr{"Ireland.csv"}, sep = \hlstr{""})), key = \hlstr{"A30"})
\end{alltt}


{\ttfamily\noindent\color{warningcolor}{\#\# Warning: cannot open file 'D:\textbackslash{}Data/data\_NFSPanelAnalysis/../data\_FADNPanelAnalysis/OrigData/eupanel9907/csv/Ireland.csv': No such file or directory}}

{\ttfamily\noindent\bfseries\color{errorcolor}{\#\# Error: cannot open the connection}}\begin{alltt}
fadn <- fadn[A30 == 4110]
\end{alltt}


{\ttfamily\noindent\bfseries\color{errorcolor}{\#\# Error: object 'fadn' not found}}\begin{alltt}
\hlkwd{setkey}(fadn, Farm.Code, YEAR)  \hlcom{# reset key}
\end{alltt}


{\ttfamily\noindent\bfseries\color{errorcolor}{\#\# Error: object 'fadn' not found}}\begin{alltt}
fadn.farms <- fadn[, \hlkwd{length}(Farm.Code), by = YEAR]
\end{alltt}


{\ttfamily\noindent\bfseries\color{errorcolor}{\#\# Error: object 'fadn' not found}}\end{kframe}
\end{knitrout}


\subsection*{Comparison of datasets}
\subsubsection*{Farm numbers by year}
Compare farm numbers to choose an easier year for individual variable comparison... 

\begin{kframe}


{\ttfamily\noindent\bfseries\color{errorcolor}{\#\# Error: object 'fadn.farms' not found}}\end{kframe}


... which shows that the years 2004 and 2005 were exactly the same size as unweighted samples. Gives some confidence that there's no difference in the system code (and there shouldn't be any difference). 

FADN weights differ from NFS weights, so I can't compare weighted samples. 

I'll select one year to compare, say 2005,  as I'd guess the two should be most similar in the most recent year that matches up the sampled farms. 

\textbf{nidy_means.Rnw} calculated means for all variables in each dataset, stored them two vectors, and compared them.

This time I'll do the comparison on an element by element basis \emph{i.e.} the individual values of each variable.

\begin{knitrout}
\definecolor{shadecolor}{rgb}{0.969, 0.969, 0.969}\color{fgcolor}\begin{kframe}
\begin{alltt}
\hlcom{# create two new data tables, with varnames and mean values}
nfs05 <- nfs[year == 2005]
fadn05 <- fadn[YEAR == 2005]
\end{alltt}


{\ttfamily\noindent\bfseries\color{errorcolor}{\#\# Error: object 'fadn' not found}}\begin{alltt}

\hlcom{# round off the means in hopes of getting exact matches this way nfs05[,}
\hlcom{# Means:=round(Means)] fadn05[, Means:=round(Means)]}

\hlcom{# set the key for each to the farm size variables in each dataset (which I}
\hlcom{# think match up)}
\hlkwd{setkey}(nfs05, fsizuaa)
\hlkwd{setkey}(fadn05, Total.UAA)
\end{alltt}


{\ttfamily\noindent\bfseries\color{errorcolor}{\#\# Error: object 'fadn05' not found}}\begin{alltt}

\end{alltt}
\end{kframe}
\end{knitrout}


That obviously didn't work... Have to review my merging with data.table.

\begin{knitrout}
\definecolor{shadecolor}{rgb}{0.969, 0.969, 0.969}\color{fgcolor}\begin{kframe}
\begin{alltt}
\hlcom{# create a logical vector for the comparison of total uaa}
nfs05$in.range <- nfs05$fsizuaa >= 0.99 * fadn05$Total.UAA & nfs05$fsizuaa <= 
    1.01 * fadn05$Total.UAA
\end{alltt}


{\ttfamily\noindent\bfseries\color{errorcolor}{\#\# Error: object 'fadn05' not found}}\begin{alltt}
\hlcom{# copy the UAA variable and fadn farmcode from the fadn to the nfs data}
nfs05$Total.UAA <- fadn05$Total.UAA
\end{alltt}


{\ttfamily\noindent\bfseries\color{errorcolor}{\#\# Error: object 'fadn05' not found}}\begin{alltt}
nfs05$Farm.Code <- fadn05$Farm.Code
\end{alltt}


{\ttfamily\noindent\bfseries\color{errorcolor}{\#\# Error: object 'fadn05' not found}}\begin{alltt}
\hlcom{# have a look at the data to double check it's what we expect}
\hlkwd{View}(nfs05[in.range == TRUE, \hlkwd{list}(in.range, farmcode, fsizuaa, Farm.Code, Total.UAA)])
\end{alltt}


{\ttfamily\noindent\bfseries\color{errorcolor}{\#\# Error: object 'in.range' not found}}\begin{alltt}
\hlcom{# viewData(nfs05[in.range==TRUE,list(in.range, farmcode, fsizuaa,}
\hlcom{# Farm.Code, Total.UAA)]) # for viewing data within the GUI if you are}
\hlcom{# using the most recent version of RStudio}

\hlcom{# count of farmcodes with matching UAA in both datasets}
\hlkwd{length}(nfs05[in.range == TRUE, farmcode])
\end{alltt}


{\ttfamily\noindent\bfseries\color{errorcolor}{\#\# Error: object 'in.range' not found}}\begin{alltt}

\hlcom{# select one of these farmcodes to compare variable by variable for}
\hlcom{# potential variable matches (creating a key).}
nfs.in.range <- nfs05[in.range == TRUE]
\end{alltt}


{\ttfamily\noindent\bfseries\color{errorcolor}{\#\# Error: object 'in.range' not found}}\begin{alltt}
nfs.farm <- nfs.in.range[1]
\end{alltt}


{\ttfamily\noindent\bfseries\color{errorcolor}{\#\# Error: object 'nfs.in.range' not found}}\begin{alltt}
\hlkwd{length}(\hlkwd{names}(nfs.farm))  \hlcom{# number of variables starting out}
\end{alltt}


{\ttfamily\noindent\bfseries\color{errorcolor}{\#\# Error: object 'nfs.farm' not found}}\begin{alltt}
nums <- nfs.farm[, \hlkwd{sapply}(.SD, is.numeric)]  \hlcom{# numeric variables}
\end{alltt}


{\ttfamily\noindent\bfseries\color{errorcolor}{\#\# Error: object 'nfs.farm' not found}}\begin{alltt}
nfs.farm <- nfs.farm[, nums == T, with = F]  \hlcom{# select only those variables}
\end{alltt}


{\ttfamily\noindent\bfseries\color{errorcolor}{\#\# Error: object 'nfs.farm' not found}}\begin{alltt}
\hlkwd{length}(\hlkwd{names}(nfs.farm))  \hlcom{# number of variables after selection}
\end{alltt}


{\ttfamily\noindent\bfseries\color{errorcolor}{\#\# Error: object 'nfs.farm' not found}}\begin{alltt}
over0 <- nfs.farm[, \hlkwd{sapply}(.SD, \hlkwd{function}(x) x > 0)]  \hlcom{# variables with value > 0}
\end{alltt}


{\ttfamily\noindent\bfseries\color{errorcolor}{\#\# Error: object 'nfs.farm' not found}}\begin{alltt}
nfs.farm <- nfs.farm[, over0 == T, with = F]  \hlcom{# select only those variables}
\end{alltt}


{\ttfamily\noindent\bfseries\color{errorcolor}{\#\# Error: object 'nfs.farm' not found}}\begin{alltt}
\hlkwd{length}(\hlkwd{names}(nfs.farm))  \hlcom{# number of variables after selection}
\end{alltt}


{\ttfamily\noindent\bfseries\color{errorcolor}{\#\# Error: object 'nfs.farm' not found}}\begin{alltt}

\hlcom{# store the corresponding farm ID's for each dataset}
farmcode.sel <- nfs.farm[, farmcode]
\end{alltt}


{\ttfamily\noindent\bfseries\color{errorcolor}{\#\# Error: object 'nfs.farm' not found}}\begin{alltt}
Farm.Code.sel <- nfs.farm[, Farm.Code]
\end{alltt}


{\ttfamily\noindent\bfseries\color{errorcolor}{\#\# Error: object 'nfs.farm' not found}}\begin{alltt}


\hlcom{# extract corresponding row from FADN}
fadn.farm <- fadn05[Farm.Code == Farm.Code.sel]
\end{alltt}


{\ttfamily\noindent\bfseries\color{errorcolor}{\#\# Error: object 'fadn05' not found}}\begin{alltt}
\hlkwd{length}(\hlkwd{names}(fadn.farm))  \hlcom{# number of variables starting out}
\end{alltt}


{\ttfamily\noindent\bfseries\color{errorcolor}{\#\# Error: object 'fadn.farm' not found}}\begin{alltt}
nums <- fadn.farm[, \hlkwd{sapply}(.SD, is.numeric)]  \hlcom{# numeric variables}
\end{alltt}


{\ttfamily\noindent\bfseries\color{errorcolor}{\#\# Error: object 'fadn.farm' not found}}\begin{alltt}
fadn.farm <- fadn.farm[, nums == T, with = F]
\end{alltt}


{\ttfamily\noindent\bfseries\color{errorcolor}{\#\# Error: object 'fadn.farm' not found}}\begin{alltt}
\hlkwd{length}(\hlkwd{names}(fadn.farm))  \hlcom{# number of variables after selection}
\end{alltt}


{\ttfamily\noindent\bfseries\color{errorcolor}{\#\# Error: object 'fadn.farm' not found}}\begin{alltt}
over0 <- fadn.farm[, \hlkwd{sapply}(.SD, \hlkwd{function}(x) x > 0)]  \hlcom{# variables with value > 0}
\end{alltt}


{\ttfamily\noindent\bfseries\color{errorcolor}{\#\# Error: object 'fadn.farm' not found}}\begin{alltt}
fadn.farm <- fadn.farm[, over0 == T, with = F]
\end{alltt}


{\ttfamily\noindent\bfseries\color{errorcolor}{\#\# Error: object 'fadn.farm' not found}}\begin{alltt}
\hlkwd{length}(\hlkwd{names}(fadn.farm))  \hlcom{# number of variables after selection}
\end{alltt}


{\ttfamily\noindent\bfseries\color{errorcolor}{\#\# Error: object 'fadn.farm' not found}}\begin{alltt}


\hlcom{# transpose both and search fadn column for possible matches for each}
\hlcom{# value of the NFS column nfs.farm <-}
\hlcom{# data.table(na.omit(data.frame(nfs.names = names(nfs.farm),}
\hlcom{# values=t(nfs.farm), row.names=NULL, stringsAsFactors=F)), key='values')}

fadn.farm <- \hlkwd{data.table}(\hlkwd{na.omit}(\hlkwd{data.frame}(fadn.names = \hlkwd{names}(fadn.farm), values = \hlkwd{t}(fadn.farm), 
    row.names = NULL, stringsAsFactors = F)), key = \hlstr{"values"})
\end{alltt}


{\ttfamily\noindent\bfseries\color{errorcolor}{\#\# Error: object 'fadn.farm' not found}}\begin{alltt}

\hlcom{# creates a logical matrix for potential matches}
key <- nfs.farm[, \hlkwd{sapply}(.SD, \hlkwd{function}(x) \hlkwd{ifelse}(x >= 0.98 * fadn.farm$values & 
    x <= 1.02 * fadn.farm$values, fadn.farm$fadn.name, NA))]  \hlcom{# ], fadn.names=fadn.farm$fadn.names),by=nfs.names]}
\end{alltt}


{\ttfamily\noindent\bfseries\color{errorcolor}{\#\# Error: object 'nfs.farm' not found}}\begin{alltt}

\hlcom{# now collapse that to a list of names of fadn variable names for each nfs}
\hlcom{# variable name}

key2 <- \hlkwd{data.table}(key)
key2 <- key2[, \hlkwd{sapply}(.SD, \hlkwd{function}(x) \hlkwd{list}(\hlkwd{na.omit}(x)))]
\hlkwd{viewData}(key2)
\end{alltt}


{\ttfamily\noindent\bfseries\color{errorcolor}{\#\# Error: could not find function "{}viewData"{}}}\begin{alltt}


\hlkwd{save}(key2, file = \hlstr{"key2.Rdata"})
\end{alltt}
\end{kframe}
\end{knitrout}


The output gives a data table with nfs varnames as the header row and fadn varnames that possibly match in the values. This data structure must have equal column lengths, so the lists of fadn names are repeated until all column lengths are the same. Nonetheless, this is a much more compact way of browsing the likely matches. 

Unfortunately, this technique did not prove effective. I'm now giving up this approach as I don't think that it will ever be fruitful in any sort of reasonable time. I will simply hardcode from the list Brian gave me, and from hard graft.

\end{flushleft}
\end{document}
